\documentclass[]{article}
\usepackage{lmodern}
\usepackage{amssymb,amsmath}
\usepackage{ifxetex,ifluatex}
\usepackage{fixltx2e} % provides \textsubscript
\ifnum 0\ifxetex 1\fi\ifluatex 1\fi=0 % if pdftex
  \usepackage[T1]{fontenc}
  \usepackage[utf8]{inputenc}
\else % if luatex or xelatex
  \ifxetex
    \usepackage{mathspec}
  \else
    \usepackage{fontspec}
  \fi
  \defaultfontfeatures{Ligatures=TeX,Scale=MatchLowercase}
\fi
% use upquote if available, for straight quotes in verbatim environments
\IfFileExists{upquote.sty}{\usepackage{upquote}}{}
% use microtype if available
\IfFileExists{microtype.sty}{%
\usepackage{microtype}
\UseMicrotypeSet[protrusion]{basicmath} % disable protrusion for tt fonts
}{}
\usepackage[margin=1in]{geometry}
\usepackage{hyperref}
\hypersetup{unicode=true,
            pdftitle={Digit-memorization},
            pdfauthor={Ellisiv Sætherø Steen, Johanne Skogvang},
            pdfborder={0 0 0},
            breaklinks=true}
\urlstyle{same}  % don't use monospace font for urls
\usepackage{color}
\usepackage{fancyvrb}
\newcommand{\VerbBar}{|}
\newcommand{\VERB}{\Verb[commandchars=\\\{\}]}
\DefineVerbatimEnvironment{Highlighting}{Verbatim}{commandchars=\\\{\}}
% Add ',fontsize=\small' for more characters per line
\usepackage{framed}
\definecolor{shadecolor}{RGB}{248,248,248}
\newenvironment{Shaded}{\begin{snugshade}}{\end{snugshade}}
\newcommand{\KeywordTok}[1]{\textcolor[rgb]{0.13,0.29,0.53}{\textbf{#1}}}
\newcommand{\DataTypeTok}[1]{\textcolor[rgb]{0.13,0.29,0.53}{#1}}
\newcommand{\DecValTok}[1]{\textcolor[rgb]{0.00,0.00,0.81}{#1}}
\newcommand{\BaseNTok}[1]{\textcolor[rgb]{0.00,0.00,0.81}{#1}}
\newcommand{\FloatTok}[1]{\textcolor[rgb]{0.00,0.00,0.81}{#1}}
\newcommand{\ConstantTok}[1]{\textcolor[rgb]{0.00,0.00,0.00}{#1}}
\newcommand{\CharTok}[1]{\textcolor[rgb]{0.31,0.60,0.02}{#1}}
\newcommand{\SpecialCharTok}[1]{\textcolor[rgb]{0.00,0.00,0.00}{#1}}
\newcommand{\StringTok}[1]{\textcolor[rgb]{0.31,0.60,0.02}{#1}}
\newcommand{\VerbatimStringTok}[1]{\textcolor[rgb]{0.31,0.60,0.02}{#1}}
\newcommand{\SpecialStringTok}[1]{\textcolor[rgb]{0.31,0.60,0.02}{#1}}
\newcommand{\ImportTok}[1]{#1}
\newcommand{\CommentTok}[1]{\textcolor[rgb]{0.56,0.35,0.01}{\textit{#1}}}
\newcommand{\DocumentationTok}[1]{\textcolor[rgb]{0.56,0.35,0.01}{\textbf{\textit{#1}}}}
\newcommand{\AnnotationTok}[1]{\textcolor[rgb]{0.56,0.35,0.01}{\textbf{\textit{#1}}}}
\newcommand{\CommentVarTok}[1]{\textcolor[rgb]{0.56,0.35,0.01}{\textbf{\textit{#1}}}}
\newcommand{\OtherTok}[1]{\textcolor[rgb]{0.56,0.35,0.01}{#1}}
\newcommand{\FunctionTok}[1]{\textcolor[rgb]{0.00,0.00,0.00}{#1}}
\newcommand{\VariableTok}[1]{\textcolor[rgb]{0.00,0.00,0.00}{#1}}
\newcommand{\ControlFlowTok}[1]{\textcolor[rgb]{0.13,0.29,0.53}{\textbf{#1}}}
\newcommand{\OperatorTok}[1]{\textcolor[rgb]{0.81,0.36,0.00}{\textbf{#1}}}
\newcommand{\BuiltInTok}[1]{#1}
\newcommand{\ExtensionTok}[1]{#1}
\newcommand{\PreprocessorTok}[1]{\textcolor[rgb]{0.56,0.35,0.01}{\textit{#1}}}
\newcommand{\AttributeTok}[1]{\textcolor[rgb]{0.77,0.63,0.00}{#1}}
\newcommand{\RegionMarkerTok}[1]{#1}
\newcommand{\InformationTok}[1]{\textcolor[rgb]{0.56,0.35,0.01}{\textbf{\textit{#1}}}}
\newcommand{\WarningTok}[1]{\textcolor[rgb]{0.56,0.35,0.01}{\textbf{\textit{#1}}}}
\newcommand{\AlertTok}[1]{\textcolor[rgb]{0.94,0.16,0.16}{#1}}
\newcommand{\ErrorTok}[1]{\textcolor[rgb]{0.64,0.00,0.00}{\textbf{#1}}}
\newcommand{\NormalTok}[1]{#1}
\usepackage{graphicx,grffile}
\makeatletter
\def\maxwidth{\ifdim\Gin@nat@width>\linewidth\linewidth\else\Gin@nat@width\fi}
\def\maxheight{\ifdim\Gin@nat@height>\textheight\textheight\else\Gin@nat@height\fi}
\makeatother
% Scale images if necessary, so that they will not overflow the page
% margins by default, and it is still possible to overwrite the defaults
% using explicit options in \includegraphics[width, height, ...]{}
\setkeys{Gin}{width=\maxwidth,height=\maxheight,keepaspectratio}
\IfFileExists{parskip.sty}{%
\usepackage{parskip}
}{% else
\setlength{\parindent}{0pt}
\setlength{\parskip}{6pt plus 2pt minus 1pt}
}
\setlength{\emergencystretch}{3em}  % prevent overfull lines
\providecommand{\tightlist}{%
  \setlength{\itemsep}{0pt}\setlength{\parskip}{0pt}}
\setcounter{secnumdepth}{0}
% Redefines (sub)paragraphs to behave more like sections
\ifx\paragraph\undefined\else
\let\oldparagraph\paragraph
\renewcommand{\paragraph}[1]{\oldparagraph{#1}\mbox{}}
\fi
\ifx\subparagraph\undefined\else
\let\oldsubparagraph\subparagraph
\renewcommand{\subparagraph}[1]{\oldsubparagraph{#1}\mbox{}}
\fi

%%% Use protect on footnotes to avoid problems with footnotes in titles
\let\rmarkdownfootnote\footnote%
\def\footnote{\protect\rmarkdownfootnote}

%%% Change title format to be more compact
\usepackage{titling}

% Create subtitle command for use in maketitle
\newcommand{\subtitle}[1]{
  \posttitle{
    \begin{center}\large#1\end{center}
    }
}

\setlength{\droptitle}{-2em}

  \title{Digit-memorization}
    \pretitle{\vspace{\droptitle}\centering\huge}
  \posttitle{\par}
    \author{Ellisiv Sætherø Steen, Johanne Skogvang}
    \preauthor{\centering\large\emph}
  \postauthor{\par}
      \predate{\centering\large\emph}
  \postdate{\par}
    \date{05/03 2019}


\begin{document}
\maketitle

\section{Introduction}\label{introduction}

Memory is one of the most important features of the human brain. Without
it, humans would not be able to learn anything, and it is necessary for
the daily functionality. An important part of the memory is the
short-term memory, which is the brains capacity to remeber a small
amount of information for a small period of time. If we know how to
utilize our capacity in the best possible way, we will be able to
remember more efficiently and hopefully

From before, we know that the short-term memory has about the capacity
to remember seven, plus or minus two, items at the same time. Our goal
is to test wheter this holds or not.

\section{Selection of factors and
response}\label{selection-of-factors-and-response}

The first thing we tested was to memorize words. A list of wors was
given to the test person, from which she tried to memorize as many words
as possible. This was tested with different factors, and two different
persons. Each time the number of words memorized, without exception, was
six. Hence, we decided to give the same experiment a try, only
exchanging worsd with digits. This resulted in a much higher variance
than when we tried memorizing words. The factors we think are relevant
when testing the short term memory is the amount of time the test person
gets to look at the digits, wheter the person is disturbed between
memorizing and writing down the digits or not and wheter she is allowed
to use aids, such as pen and paper, while looking at the digits. We
expect an interraction between time and aid, thinking that they will
amplify each other. For each of the factors we think that two levels
will be the most suitable, i.e.~short and long time to look at the
digits, disturbance or not between looking at the digits and writing the
down and wheter she gets to use aids or not. One could argue for three
levelsof time, but we believe that two levels will be suficient if one
for example double the amount of time. Also, we think that it is easy to
control that the factors are at the desired level since we can measure
the time, choose to use aids or not and we can make sure of disturbance
or not our self.

A response variable that will give the information we need is the number
of digits the test person remembers in the right order until the first
mistake, meaning that if the first digit is wrong, you get Y equal to
zero, regardless of the correctness of the next digits. In theory the
test person gets an infinite number of digits, but since time is
limited, one can also limit the number of digits given, and still count
it as infinite. One could also chose to get a finite number of digits or
numbers, and count the amount that is in the right spot. There are also
a lot of different variations one could do when counting the amount og
numbers that the test person are able to memorize, the most important
thing is that one are consistent. As mentioned, one could also count
words instead of digits. The accuracy when using the first response
variable is 100\% since we easily can count how many digits there are
until the first mistake.

\section{Choice of design}\label{choice-of-design}

We choose a full \(2^k\) factorial experiments with replicate because of
the simple nature of the experiment. One full trial does not need
complicated equipment and replication causes minimal resources. The only
problem with doing multiple experiments is that the test person gets
tired and it is hard to reset between rounds. Because of that we choose
to use two different test people and do one full trial with both of
them. The result is a \(2^4\) factorial design with \(2^5\) experiments.
Because of the different test people it is necessary to divide the
design into two blocks, one block representing a test person. This is
needed because the memory of the test people can differ, so the block
factor needs to compensate for this.

\section{Experimental set-up}\label{experimental-set-up}

We chose the factors discussed above, but the value of the two levels
needs to be determined and we need to choose some guidelines to make the
experiments consistent. The two time intervals needs to be sufficiently
different, but at the same time we do not want to have too long time
intervals to cause as little as possible exhaustion for the test
persons. We decide on the short interval, coded as \(time = -1\), to be
\(30\)sec and the long time interval, coded as \(time = 1\), to be
\(60\)sec.~The aids provided is pen and paper and the possibility of
speaking out loud. To not make this a writing contest the paper is taken
away from the test person after the time for memorizing is up. To make
the task a bit more challenging the test people get 15 seconds break
between momorizing and writing down their final answer. This time is
constant and is thus not a factor in our trials. It is also in this
period the person could be disturbed. After some discussion we choose
that there should just be loud talking in the background which needs no
respons from the test person, or else it would be too hard to remember
anything.\\
To explore the quality of our experiments, we include a variable which
we assume will have no effect on the result, which is if the test person
is wearing shoes. Because we do a limited amount of experiments it is
interessting to see if the random variable will be included in the final
model we choose. To minimize the influence of the experimental order and
to exlude time as a factor we randomize the order of the experiments for
both trials.\\
There are several challenges when measuring the response variable. First
of all the digits needs to be totally random and different every time.
To solve this we use the functionality of the web-page Wolfram-Alpha.
For every test run it generates a new random number with 40 digits which
makes the test runs independent. Another challenge is the measurement of
the quality of the response. What if the first digits are wrong, but
others are rightly placed? And what if a digit is excluded and that
shifts the order of the rest of the answer? To make the rules simple and
consistent, we then decided the response, \(y\), to be the number of
digits rightly guessed up to the first mistake.

\section{Results}\label{results}

\begin{Shaded}
\begin{Highlighting}[]
\CommentTok{#install.packages('FrF2')}
\KeywordTok{library}\NormalTok{(FrF2)}
\end{Highlighting}
\end{Shaded}

\begin{verbatim}
## Warning: package 'FrF2' was built under R version 3.5.3
\end{verbatim}

\begin{verbatim}
## Loading required package: DoE.base
\end{verbatim}

\begin{verbatim}
## Warning: package 'DoE.base' was built under R version 3.5.3
\end{verbatim}

\begin{verbatim}
## Loading required package: grid
\end{verbatim}

\begin{verbatim}
## Loading required package: conf.design
\end{verbatim}

\begin{verbatim}
## 
## Attaching package: 'DoE.base'
\end{verbatim}

\begin{verbatim}
## The following objects are masked from 'package:stats':
## 
##     aov, lm
\end{verbatim}

\begin{verbatim}
## The following object is masked from 'package:graphics':
## 
##     plot.design
\end{verbatim}

\begin{verbatim}
## The following object is masked from 'package:base':
## 
##     lengths
\end{verbatim}

\begin{Shaded}
\begin{Highlighting}[]
\NormalTok{plan <-}\StringTok{ }\KeywordTok{FrF2}\NormalTok{(}\DataTypeTok{nruns =} \DecValTok{32}\NormalTok{, }\DataTypeTok{nfactors=}\DecValTok{5}\NormalTok{, }\DataTypeTok{randomize =}\NormalTok{ F)}
\end{Highlighting}
\end{Shaded}

\begin{verbatim}
## creating full factorial with 32 runs ...
\end{verbatim}

\begin{Shaded}
\begin{Highlighting}[]
\NormalTok{y <-}\StringTok{ }\KeywordTok{c}\NormalTok{(}\DecValTok{9}\NormalTok{, }\DecValTok{9}\NormalTok{, }\DecValTok{9}\NormalTok{, }\DecValTok{4}\NormalTok{, }\DecValTok{6}\NormalTok{, }\DecValTok{15}\NormalTok{, }\DecValTok{6}\NormalTok{, }\DecValTok{3}\NormalTok{, }\DecValTok{11}\NormalTok{, }\DecValTok{15}\NormalTok{, }\DecValTok{12}\NormalTok{, }\DecValTok{12}\NormalTok{, }\DecValTok{5}\NormalTok{, }\DecValTok{12}\NormalTok{, }\DecValTok{6}\NormalTok{, }\DecValTok{1}\NormalTok{, }\DecValTok{9}\NormalTok{, }\DecValTok{15}\NormalTok{, }\DecValTok{9}\NormalTok{, }\DecValTok{11}\NormalTok{, }\DecValTok{12}\NormalTok{, }\DecValTok{12}\NormalTok{, }\DecValTok{6}\NormalTok{, }\DecValTok{12}\NormalTok{, }\DecValTok{12}\NormalTok{, }\DecValTok{0}\NormalTok{, }\DecValTok{9}\NormalTok{, }\DecValTok{15}\NormalTok{, }\DecValTok{7}\NormalTok{, }\DecValTok{15}\NormalTok{, }\DecValTok{6}\NormalTok{, }\DecValTok{9}\NormalTok{)}
\NormalTok{plan <-}\StringTok{ }\KeywordTok{add.response}\NormalTok{(plan, y)}
\CommentTok{#plan <- plan[, -4]}
\NormalTok{lm4 <-}\StringTok{ }\KeywordTok{lm}\NormalTok{(y}\OperatorTok{~}\NormalTok{(A }\OperatorTok{+}\StringTok{ }\NormalTok{B }\OperatorTok{+}\StringTok{ }\NormalTok{C }\OperatorTok{+}\StringTok{ }\NormalTok{D)}\OperatorTok{^}\DecValTok{4} \OperatorTok{+}\StringTok{ }\NormalTok{E, }\DataTypeTok{data=}\NormalTok{plan)}
\KeywordTok{summary}\NormalTok{(lm4)}
\end{Highlighting}
\end{Shaded}

\begin{verbatim}
## 
## Call:
## lm.default(formula = y ~ (A + B + C + D)^4 + E, data = plan)
## 
## Residuals:
##    Min     1Q Median     3Q    Max 
## -8.250 -1.125  0.000  1.125  8.250 
## 
## Coefficients:
##               Estimate Std. Error t value Pr(>|t|)    
## (Intercept)  9.188e+00  7.188e-01  12.782 1.82e-09 ***
## A1           8.125e-01  7.188e-01   1.130    0.276    
## B1          -1.063e+00  7.188e-01  -1.478    0.160    
## C1          -8.750e-01  7.188e-01  -1.217    0.242    
## D1          -5.275e-17  7.188e-01   0.000    1.000    
## E1           7.500e-01  7.188e-01   1.043    0.313    
## A1:B1       -5.625e-01  7.188e-01  -0.783    0.446    
## A1:C1        7.500e-01  7.188e-01   1.043    0.313    
## A1:D1       -1.250e-01  7.188e-01  -0.174    0.864    
## B1:C1       -1.125e+00  7.188e-01  -1.565    0.138    
## B1:D1        6.250e-01  7.188e-01   0.870    0.398    
## C1:D1       -6.875e-01  7.188e-01  -0.956    0.354    
## A1:B1:C1    -8.750e-01  7.188e-01  -1.217    0.242    
## A1:B1:D1     3.750e-01  7.188e-01   0.522    0.609    
## A1:C1:D1     1.875e-01  7.188e-01   0.261    0.798    
## B1:C1:D1    -5.625e-01  7.188e-01  -0.783    0.446    
## A1:B1:C1:D1 -1.062e+00  7.188e-01  -1.478    0.160    
## ---
## Signif. codes:  0 '***' 0.001 '**' 0.01 '*' 0.05 '.' 0.1 ' ' 1
## 
## Residual standard error: 4.066 on 15 degrees of freedom
## Multiple R-squared:  0.5239, Adjusted R-squared:  0.01601 
## F-statistic: 1.032 on 16 and 15 DF,  p-value: 0.4782
\end{verbatim}

The results of the experiment is shown above in the summary.\\
A1 = Long time for memorization\\
B1 = Available aids\\
C1 = Presence of disturbance\\
D1 = Wearing shoes E = Block factor/test person\\
For a \(2^k\) factorial experiment, we are doing inference on the main
effects, \(2\beta_j\). The resulting model is thus

\[\hat{y_i} = 9.19 + 0.81A - 1.06B - 0.88C + 0D + 0.75E - 0.56AB + 0.75AC - 0.12AD  - 1.12BC + 0.62BD-0.69CD- 0.87ABC + 0.37ABD+ 0.19ACD- 0.56BCD- 1.06ABCD\]
We see from the p-values in the summary that none of the factors are
characterized to be non-zero with a certainty of more than 85\%, wich is
not good at all. We inspect the residual plot to see if this can give
more insight in the quality of our measurements.

\subsection{Experimental weaknesses}\label{experimental-weaknesses}

\begin{Shaded}
\begin{Highlighting}[]
\NormalTok{rres <-}\StringTok{ }\KeywordTok{rstudent}\NormalTok{(lm4)}
\KeywordTok{plot}\NormalTok{(lm4}\OperatorTok{$}\NormalTok{fitted,rres, }\KeywordTok{title}\NormalTok{(}\StringTok{"Studentized residual plot"}\NormalTok{))}
\end{Highlighting}
\end{Shaded}

\includegraphics{Linstat-strikk_files/figure-latex/residual plot-1.pdf}

\begin{Shaded}
\begin{Highlighting}[]
\KeywordTok{qqnorm}\NormalTok{(rres)}
\KeywordTok{qqline}\NormalTok{(rres)}
\end{Highlighting}
\end{Shaded}

\includegraphics{Linstat-strikk_files/figure-latex/residual plot-2.pdf}

\begin{Shaded}
\begin{Highlighting}[]
\KeywordTok{library}\NormalTok{(nortest)}
\KeywordTok{ad.test}\NormalTok{(}\KeywordTok{rstudent}\NormalTok{(lm4))}
\end{Highlighting}
\end{Shaded}

\begin{verbatim}
## 
##  Anderson-Darling normality test
## 
## data:  rstudent(lm4)
## A = 1.6254, p-value = 0.0002824
\end{verbatim}

We look at the residual plot and the Anderson-Darling test and see that
our residuals does not look very normally distributed. This is probably
caused by some very unlikely obeservations, like the two outliers
visible in the residual plot. We therefore discuss what went wrong in
the experiment to try to find an explaination. The implementation went
more or less as we planned. It happened in the first trial that we read
the randomized design matrix wrong and thus switched the order of two of
the tests. As we see it, this did not affect the trial because the order
was random and thus a random switch will not lead to a loss of
randomization. We also observed that some sequences of digits were
easier to remember than others. Also, the number generator had an output
of blocks of three digits, and thus there were an overrepresentation in
the responses of numbers dividable of three. It is also possible that
the blocks made it easier to remember. These kind of factors could have
lead to some extreme observations. Additionally we had some problems
because of our simple measurement of response. One time the test person
failed to remember the first digit, whih lead to \(y=0\) even though the
person remembered many other digits right. This is a major flaw in our
experiment because if the test person does an early mistake it does not
need to say much abouth his or her memory.\\
\#Also the experiment was done in unpredictable surroundings, where
people walked by and talked among other things. It was therefore a bit
variety in the disturbance-factor, but it was not much difference and
because of the random order, it is not unreasonable to assume that this
did not affect the result much. We conclude that the observation that
felt most problematic during the trials were observation \(y_{26}=0\).
By inspecting our residual plot we find that two observations deviate
significantly from the rest. Below we find that observation \(y_{26}\)
is one of those outliers, more specifically the point with fitted value
around 8 and residual -4.

\begin{Shaded}
\begin{Highlighting}[]
\KeywordTok{cat}\NormalTok{(}\StringTok{"Fitted value of observation 26: "}\NormalTok{, lm4}\OperatorTok{$}\NormalTok{fitted.values[}\DecValTok{26}\NormalTok{], }\StringTok{". Studentized residual: "}\NormalTok{, rres[}\DecValTok{26}\NormalTok{])}
\end{Highlighting}
\end{Shaded}

\begin{verbatim}
## Fitted value of observation 26:  8.25 . Studentized residual:  -4.446832
\end{verbatim}

\subsection{Modified results}\label{modified-results}

To remove the problematic outlier we have chosen to imputate the
observation using mean substitution. We set \(y_{26}=\text{mean}(y)\)
and fit a new model. The nice things about using mean substitution is
that unlike removing the observation we still get the properties of a
two-factor experiment. The main problem with doing this is that it
provides a kind of false sense of security and we can not trust the
results as much as the it looks like. It will also very likely lead to
smooth out the effects of the factors. This is obvious if we use the
same technique on more observations, and ultimately for all, then no
factors would have any effect.

This model provides more significant results and we observe from the
Anderson-Darling test and the studentized residual plot that the
residuals also look more like random noise. We choose to move forward
with the new model because as we discussed above, an early mistake does
not need to mean much about the persons memory, but we also have in mind
that we have manipulated our data which will influence our results.
Moving on, we analyze the new effects and use the fact that the standard
deviation of the model is t-distributed to find the significant effects
with significance level \(\alpha = 0.05\).

\begin{Shaded}
\begin{Highlighting}[]
\NormalTok{effects <-}\StringTok{ }\DecValTok{2}\OperatorTok{*}\NormalTok{modified}\OperatorTok{$}\NormalTok{coeff}
\NormalTok{effects}
\end{Highlighting}
\end{Shaded}

\begin{verbatim}
## (Intercept)          A1          B1          C1          D1          E1 
##  18.9492188   2.1992188  -2.6992188  -2.3242188   0.5742187   2.0742188 
##       A1:B1       A1:C1       A1:D1       B1:C1       B1:D1       C1:D1 
##  -1.6992187   0.9257813   0.3242187  -1.6757812   0.6757813  -1.9492187 
##    A1:B1:C1    A1:B1:D1    A1:C1:D1    B1:C1:D1 A1:B1:C1:D1 
##  -1.1757812   0.1757812  -0.1992187  -0.5507812  -1.5507813
\end{verbatim}

\begin{Shaded}
\begin{Highlighting}[]
\NormalTok{n <-}\StringTok{ }\KeywordTok{length}\NormalTok{(ymod)}
\NormalTok{sest <-}\StringTok{ }\KeywordTok{summary}\NormalTok{(modified)}\OperatorTok{$}\NormalTok{sigma}

\NormalTok{seffect <-}\StringTok{ }\DecValTok{2}\OperatorTok{*}\NormalTok{sest}\OperatorTok{/}\KeywordTok{sqrt}\NormalTok{(n) }\CommentTok{# SD of 2*beta-hat}

\NormalTok{signcut <-}\StringTok{ }\KeywordTok{qt}\NormalTok{(}\FloatTok{0.975}\NormalTok{,}\DataTypeTok{df=}\NormalTok{n}\OperatorTok{-}\DecValTok{2}\OperatorTok{^}\DecValTok{4}\NormalTok{)}\OperatorTok{*}\NormalTok{seffect}

\KeywordTok{barplot}\NormalTok{(}\KeywordTok{sort}\NormalTok{(}\KeywordTok{abs}\NormalTok{(effects[}\OperatorTok{-}\DecValTok{1}\NormalTok{]),}\DataTypeTok{decreasing=}\OtherTok{FALSE}\NormalTok{),}\DataTypeTok{las=}\DecValTok{1}\NormalTok{,}\DataTypeTok{horiz=}\OtherTok{TRUE}\NormalTok{)}
\KeywordTok{abline}\NormalTok{(}\DataTypeTok{v=}\NormalTok{signcut,}\DataTypeTok{col=}\DecValTok{2}\NormalTok{,}\DataTypeTok{lwd=}\DecValTok{2}\NormalTok{)}
\end{Highlighting}
\end{Shaded}

\includegraphics{Linstat-strikk_files/figure-latex/effects-1.pdf} We see
from the bar plot that the factors \(B\) and \(C\) corresponding to
``Available aids'' and ``Presence of disturbance'' is the only factors
considered significant with significance level \(\alpha = 0.05\). We
also observe that there are not any factors or interactions that really
stick out, and this reinforces the theory that the effects would be
smoothed out by the mean substitution. The resulting model is anyways

\[y_i = 9.47 - 1.35 x_\text{aids} - 1.16 x_\text{disturbance},\] where
\(x_\text{aids}, x_\text{disturbance} \in \{-1, 1\}\).

\section{Conclusion}\label{conclusion}

We have seen from the trials and the analysis afterwards that a well
planned experiments is key. The main problem in our case was the choice
of response, where we thought that the easiest choice would be good
enough. Even though it is important to use a response which is easy to
understand and measure, it looks like we could have gotten better
results by finding a more complicated, but more accurate measurement of
how many digits that were remembered. From the analysis we found that
there was mainly one observation that made inference particularly hard.
We used mean substitution on this observaton, which really improved the
residual plots and the significance of our factors. Even though we got
some results from our experiments, we will be carful when concluding
because of our poor choice of response measurements and because we
interferred with the results. Despite this, we have found some results
indicating that the use of writing down what you want to remember
actually does not help. Probably it is poor spending of time and from
the execution of the experiments it seemed like the test people
remembered the numbers better, but did not make it as far. The other
indication was that it is harder to remember if you are in a noisy
environment. This was hardly no surprize, but we had maybe thought that
this would have even bigger effect and that this would be worse than
having aids available. The other factors are harder to interprete, but
we can see in the bar plot that factor \(D\) hat little, if any
significance and this is completely consistent with our expectations.
Also we can see from the summary that the second test person remembered
a little better, and this is just expected variations in the population.
To summarize, we would recommend doing the experiments agian with a
slight change of response measurement. A good alternative is to count
all the rightly placed digits and give some kind of penalty for wrongly
placed digits. This would hopefully prevent extreme observations in the
low area and it gives a more accurate estimate of memorization. Maybe it
is also a good idea to exchange some of the factors. From our analysis
it seems rather clear that shoes has nothing to do with memory, so it
could be smart to just discard it as a facor.


\end{document}
